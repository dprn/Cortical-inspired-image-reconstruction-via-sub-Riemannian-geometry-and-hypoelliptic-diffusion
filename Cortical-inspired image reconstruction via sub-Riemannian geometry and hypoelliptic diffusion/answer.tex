\documentclass{amsart}
\usepackage{amssymb}
\usepackage{amsfonts}
\usepackage{amsmath}
\usepackage{graphicx}
\usepackage{color}


\newcommand{\dario}{\textcolor{magenta}}



%%%%%%%%%%%%%%%%%%%%%%%%%%%%%%%%%%%%%%%%
%``good'' and ``bad''
%%%%%%%%%%%%%%%%%%%%%%%%%%%%%%%%%%%%%%%%


\newcommand{\blue}{\color{blue}}
\newcommand{\red}{\color{red}}

\title[Answer to the reviewers  comments]{Answer to the reviewers comments to the paper ``Cortical-inspired image reconstruction via sub-Riemannian geometry and hypoelliptic diffusion''}

\begin{document}


\maketitle

We are very thankful to the reviewers for their valuable remarks. 
We made the corresponding corrections. 

\section*{Reviewer 2}

We corrected all typos remarked by the 2nd reviewer. 


\section*{Reviewer 1}

1. We added new references suggested by the reviewer and text 
\textcolor{blue}{(the blue paragraphs in the beginning of Introduction)} 
in order to explain what is the contributions of the proposed approaches 
with respect to existing previous works.

-- \textcolor{red}{
{\it What are the novelty? Is it supposed to be better? Faster?
} -- I have no answer, 
I have just explain why does our diffusion differ from other anisotropic diffusions described in the works 
suggested by the reviewer.
}

\medskip 

2. 
{\it More generally, I found that the paper fails to really explain what is the contributions of
the proposed approaches with respect to existing previous works. Inpainting (using
variational methods) is a huge field, and the paper does not provide links or connection
with other works. It should somehow explain more clearly, put aside mathematical rigor
and connection to geometry, what are the contributions.}

We explained the main difference between our our hypoelliptic diffusion and other anisotropic diffusions, 
we provided connection with other works known to us by adding some additional text 
\textcolor{blue}{(the blue paragraphs in the beginning of Introduction)}. 
However, in this short conference paper we do not pretend to give a survey and comparison of our algorithms with all existing methods.

\medskip 

3. 
{\it 
The numerical results are (put aside fig 9 right) restricted to the filling-in small holes
(either thin elongated holes, or random points). Does this suggest that the proposed
methods are not capable of handling large missing regions without introducing too much
blurring? 

\textcolor{red}{
Can this be predicted from the theory? -- {\rm I have no answer !} 
}

Some anisotropic diffusion methods (used for instance in [Galic, Tschumperle]) 
seem to be capable of doing so, by introducing stronger (often maybe non convex) contrast 
enhancing terms (or non variational evolution with some reaction term). 

How would this fits into the proposed framework ? In any case,
if this is a limitation of the method, it should be mentioned.
}

\medskip 

The existence of large ``holes'' (i.e., regions without noncorrupted pixels) is an obstacle for 
good quality of reconstruction for any inpainting method. For good quality of reconstruction, 
non-corrupted pixels must be well-distributed on the image, especially in regions containing 
important small details (which, in fact, can be totally disappeared after corruption). 
Our method is not an exception, and the presence of large holes (especially if they cover essential small details of the image) yields falsifications of inpainting. 
We discussed this effect in our previous paper ``Highly Corrupted Image Inpainting Through Hypoelliptic Diffusion'' (see especially Conclusion and Fig.~10).

\textcolor{blue}{
However, now we added a small text about this problem in the end of Section~4.3.
}

It is worst observing that the paper [Galic] presents reconstructions of images with well distributed 
non-corrupted pixels (Fig.~1, Fig.~4), which are comparable with ours (Fig.~14 and Fig.~15 of our paper). 
The paper [Tschumperle] contains images with another type of corruption (Fig.~6, right down), but this image 
does not contain essential small details, therefore the largest diameter of corrupted parts (white squares) 
does not yield a serious falsification of the inpainting. 


\end{document}



